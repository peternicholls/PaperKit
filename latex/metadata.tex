% ==============================================================================
% Metadata: Title, Author, Date, Abstract
% ==============================================================================

\title{Color Journey Engine: A Perceptually-Uniform Palette Generation Specification}

\author{Peter Nicholls}

\date{\today}

% ==============================================================================
% Abstract
% ==============================================================================
% Note: Replace this placeholder with your actual abstract when ready
\newcommand{\abstracttext}{%
  This paper specifies the Color Journey Engine, a deterministic algorithm for 
  generating perceptually-uniform color palettes. Built on the OKLab color space, 
  the engine uses cubic Bézier curves to construct ``journeys'' through perceptual 
  color space, producing coherent palettes from one to five anchor colors. A novel 
  single-anchor expansion algorithm---to our knowledge not previously formalised---generates 
  harmonious variations from a single color by exploiting lightness-weighted directions 
  in perceptual space. The specification defines perceptual constraints based on 
  just-noticeable difference thresholds, user-facing style controls for temperature, 
  intensity, and smoothness, and multiple loop strategies for extended sequences. 
  A two-layer gamut management approach ensures valid sRGB output while preserving 
  design intent. The engine operates as a pure function with complete determinism, 
  achieving 5.6 million colors per second in the reference implementation. This 
  specification serves as both a formal definition and implementation guide for 
  reproducible palette generation in user interfaces, data visualization, and 
  generative art applications.
}

% ==============================================================================
% Keywords (optional)
% ==============================================================================
\newcommand{\keywords}{color palette generation, OKLab, perceptual uniformity, Bézier curves, gamut mapping, deterministic algorithms}
