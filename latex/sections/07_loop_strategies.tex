% ==============================================================================
% Section 7: Loop Strategies
% ==============================================================================
% Primary Source: 05_loop_strategies.md
% Target Length: 1,000-1,500 words
% Dependencies: §3 (path construction), §4 (spacing constraints)
% ==============================================================================

% ------------------------------------------------------------------------------
% 7.1 Open Strategy
% ------------------------------------------------------------------------------
\section{Open Strategy}
\label{sec:loop-open}

% [To be written: ~200 words]
% Content: Default open path (no return)
% Key points:
%   - Default strategy
%   - Path starts at first anchor, ends at last
%   - No repetition, no wrap-around
%   - Output is linear sequence from start to end
%   - Best for: one-way progressions, timelines

% ------------------------------------------------------------------------------
% 7.2 Closed Strategy
% ------------------------------------------------------------------------------
\section{Closed Strategy}
\label{sec:loop-closed}

% [To be written: ~200 words]
% Content: Return to start color
% Key points:
%   - Path returns to starting color
%   - Creates seamless loop (last → first is smooth)
%   - Additional segment from last anchor to first
%   - Output: colors can be traversed cyclically
%   - Best for: cyclic data, continuous animations

% ------------------------------------------------------------------------------
% 7.3 Ping-Pong Strategy
% ------------------------------------------------------------------------------
\section{Ping-Pong Strategy}
\label{sec:loop-pingpong}

% [To be written: ~200 words]
% Content: Reverse at endpoints
% Key points:
%   - Path oscillates: forward, then backward, repeat
%   - No discontinuity at endpoints
%   - Output unrolled: A→B→C→B→A→B→C→...
%   - Endpoints appear once per cycle, midpoints twice
%   - Best for: back-and-forth animations

% ------------------------------------------------------------------------------
% 7.4 Möbius Strategy
% ------------------------------------------------------------------------------
\section{Möbius Strategy}
\label{sec:loop-mobius}

% [To be written: ~250 words]
% Content: Return with perceptual shift (half-twist)
% Key points:
%   - Returns to start but with transformation
%   - Like Möbius strip: one loop = half rotation
%   - Two full cycles to return to original
%   - Transformation typically in lightness or chroma
%   - Creates interesting visual patterns
%   - Best for: evolving loops, generative art

% ------------------------------------------------------------------------------
% 7.5 Phased Strategy
% ------------------------------------------------------------------------------
\section{Phased Strategy}
\label{sec:loop-phased}

% [To be written: ~200 words]
% Content: Progressive transformation each cycle
% Key points:
%   - Each cycle applies incremental transformation
%   - Phase shift in hue, lightness, or chroma
%   - Palette evolves over multiple cycles
%   - Good for long sequences with gradual drift
%   - Best for: evolving backgrounds, generative sequences

% ------------------------------------------------------------------------------
% 7.6 Output Semantics
% ------------------------------------------------------------------------------
\section{Output Semantics}
\label{sec:loop-output}

% [To be written: ~250 words]
% Content: How loops unroll into output arrays
% Key points:
%   - All strategies output flat array of N colors
%   - Loops "unrolled" into sequential list
%   - No nested structure or cycle metadata in output
%   - Caller handles cycling if needed
%   - Non-repetition: adjacent colors never identical

% Example table placeholder:
% \begin{table}[h]
%   \centering
%   \caption{Loop Strategy Output Examples (N=7, 3 anchors)}
%   \label{tab:loop-output}
% \end{table}
