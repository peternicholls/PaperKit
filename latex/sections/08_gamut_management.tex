% ==============================================================================
% Section 8: Gamut Management
% ==============================================================================
% Primary Source: 08_gamut_management.md
% Target Length: 500-1,000 words
% Dependencies: §2 (color space boundaries), §3 (control point placement)
% ==============================================================================

% ------------------------------------------------------------------------------
% Chapter Introduction
% ------------------------------------------------------------------------------
With path construction, perceptual constraints, style controls, and loop strategies established, one critical concern remains: ensuring that every color the engine generates is actually displayable. OKLab's perceptual uniformity comes at a cost---it can represent colors that no physical display can reproduce. This chapter addresses gamut management: the techniques for keeping generated colors within the bounds of real displays.

% ------------------------------------------------------------------------------
% 8.1 The Gamut Boundary Problem
% ------------------------------------------------------------------------------
\section{The Gamut Boundary Problem}
\label{sec:gamut-problem}

OKLab is a perceptually uniform color space, but it encompasses colors that lie outside any real display's capabilities~\cite{morovic2008}:

\begin{itemize}
    \item \textbf{High-chroma colours} at certain hue angles exceed sRGB
    \item \textbf{Very dark or very light saturated colours} clip
    \item \textbf{Certain hue regions} (cyan-blue, yellow-green) are particularly constrained
\end{itemize}

When constructing journeys, naive Bézier curves may traverse impossible colours. Hard clipping (clamping RGB values) creates visible discontinuities and hue shifts. Perceptual uniformity breaks down at gamut boundaries.

The fundamental constraint: all journey colours must satisfy
\begin{equation}
    \gamma(t) \in G_{\text{sRGB}} \subset \text{OKLab} \quad \forall t
\end{equation}
where $G_{\text{sRGB}}$ represents the sRGB gamut volume mapped into OKLab space.

% ------------------------------------------------------------------------------
% 8.2 Design-Time Gamut Awareness (Prevention Layer)
% ------------------------------------------------------------------------------
\section{Design-Time Gamut Awareness}
\label{sec:gamut-design}

The first layer of gamut management is \textbf{prevention}---constructing paths that stay within gamut.

When placing Bézier control points, the engine:

\begin{enumerate}
    \item \textbf{Prefers moderate chroma levels} --- Control points default to $\sim$70--80\% of maximum chroma at their hue/lightness
    
    \item \textbf{Detects candidate out-of-gamut points} --- Before finalising control points, checks sRGB validity
    
    \item \textbf{Pulls toward lower chroma} --- If out-of-gamut detected, reduces chroma while preserving hue and lightness
    
    \item \textbf{Uses gamut envelope models} --- Employs OKLCh gamut boundary approximations for efficient checking
\end{enumerate}

This proactive approach reduces the likelihood of out-of-gamut samples, preserving design intent where possible.

% ------------------------------------------------------------------------------
% 8.3 Gamut Correction (Hue-Preserving Chroma Reduction)
% ------------------------------------------------------------------------------
\section{Gamut Correction}
\label{sec:gamut-correction}

The second layer handles residual out-of-gamut colours through \textbf{soft correction} at sampling time.

For a candidate color in OKLCh $(L, C, h)$:

\begin{enumerate}
    \item Check sRGB validity via conversion
    \item If out-of-gamut, determine the maximum valid chroma at that $(L, h)$ by finding the intersection of the constant-hue ray with the gamut boundary
    \item Return the corrected color with reduced $C$, preserved $L$ and $h$
\end{enumerate}

\textit{Implementation note:} The intersection can be found via binary search, geometric approximation, or analytical methods depending on performance requirements. The specification requires only that the maximum valid chroma be determined; the algorithmic approach is an implementation detail.

\begin{designdecision}[Two-Layer Gamut Handling]
\textbf{Choice:} Gamut management uses two layers: (1) design-time prevention via gamut-aware control point placement, and (2) sample-time correction via chroma reduction with hue preservation.

\textbf{Rationale:} Post-correction alone can introduce perceptual discontinuities when many samples need large corrections. Prevention minimises corrections needed; correction handles residual cases gracefully.

\textbf{Alternatives Considered:}
\begin{itemize}
  \item \textit{Hard clipping (RGB clamping)} --- Rejected: destroys hue, creates visual artifacts
  \item \textit{HSV/HSL fallback} --- Rejected: breaks perceptual uniformity guarantee
  \item \textit{Post-processing only} --- Rejected: large corrections cause discontinuities
  \item \textit{Hue rotation to fit gamut} --- Rejected: changes color identity unacceptably
\end{itemize}

\textbf{Reference:} \S\ref{sec:gamut-design}, \S\ref{sec:hue-preservation}
\end{designdecision}

% ------------------------------------------------------------------------------
% 8.4 Hue Preservation (Correction Priority)
% ------------------------------------------------------------------------------
\section{Hue Preservation}
\label{sec:hue-preservation}

When gamut mapping is required, the engine follows a strict priority hierarchy rooted in established gamut mapping practice~\cite{morovic2008}:

\begin{enumerate}
    \item \textbf{Hue is sacred} --- Never shift hue to fit gamut
    \item \textbf{Lightness is strongly preserved} --- Only adjust if absolutely necessary
    \item \textbf{Chroma absorbs the compromise} --- Reduce saturation to fit
\end{enumerate}

This hierarchy is not arbitrary: research on gamut mapping algorithms shows that hue shifts are perceptually more objectionable than chroma reductions~\cite{morovic2008,fairchild2013}. Humans are particularly sensitive to hue changes in familiar objects (skin tones, sky, foliage), and even small hue shifts can appear ``wrong'' while equivalent chroma reductions appear merely ``less vivid.''

This preserves the journey's ``character'' (its hue story) even when display limitations require desaturation~\cite{hunt2004}. A ``sunset palette'' remains warm-hued even if some oranges must be less saturated.

\subsection*{Interaction with Style Controls}

Dynamic controls affect gamut behaviour:

\begin{center}
\begin{tabular}{ll}
\toprule
\textbf{Control} & \textbf{Gamut Implication} \\
\midrule
Chroma $> 1.0$ & Increases gamut pressure; more mapping likely \\
Chroma $< 1.0$ & Reduces gamut pressure; safer \\
Vibrancy high & Pushes mid-journey toward gamut edges \\
Lightness extreme & May limit available chroma range \\
Temperature shift & Changes which hue regions are under pressure \\
\bottomrule
\end{tabular}
\end{center}

If dynamics push colours out of gamut, chroma is the first to be moderated. Diagnostics report when dynamic settings caused gamut corrections.

\subsection*{Diagnostics}

The engine reports gamut-related information:
\begin{itemize}
    \item Number of colours requiring correction
    \item Maximum chroma reduction applied
    \item Which anchors were at gamut boundaries
\end{itemize}

This allows callers to understand why output differs from the theoretical ideal and adjust anchors or dynamics to avoid corrections.
