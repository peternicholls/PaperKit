% ==============================================================================
% Section 8: Gamut Management
% ==============================================================================
% Primary Source: 08_gamut_management.md
% Target Length: 500-1,000 words
% Dependencies: §2 (color space boundaries), §3 (control point placement)
% ==============================================================================

% ------------------------------------------------------------------------------
% 8.1 The Gamut Boundary Problem
% ------------------------------------------------------------------------------
\section{The Gamut Boundary Problem}
\label{sec:gamut-problem}

% [To be written: ~250 words]
% Content: Why OKLab paths can exit sRGB gamut
% Key points:
%   - OKLab is device-independent, sRGB is bounded
%   - Valid OKLab coordinates may have no sRGB representation
%   - High-chroma colors near gamut boundary
%   - Bézier curves can exit gamut even if anchors are in-gamut
%   - Must handle gracefully for practical output

% Gamut diagram placeholder:
% \begin{figure}[h]
%   \centering
%   % \includegraphics{gamut_boundary}
%   \caption{sRGB gamut boundary in OKLab space}
%   \label{fig:gamut}
% \end{figure}

% ------------------------------------------------------------------------------
% 8.2 Design-Time Gamut Awareness
% ------------------------------------------------------------------------------
\section{Design-Time Gamut Awareness}
\label{sec:gamut-design}

% [To be written: ~250 words]
% Content: First layer - gamut-aware path construction
% Key points:
%   - Control points placed with gamut awareness
%   - Intensity parameter scaled to avoid extreme excursions
%   - "Soft" boundaries guide but don't hard-clip
%   - Reduce likelihood of out-of-gamut samples
%   - Preserves design intent where possible

% ------------------------------------------------------------------------------
% 8.3 Sample-Time Correction
% ------------------------------------------------------------------------------
\section{Sample-Time Correction}
\label{sec:gamut-correction}

% [To be written: ~250 words]
% Content: Second layer - per-sample gamut mapping
% Key points:
%   - Check each sampled color for sRGB validity
%   - If out-of-gamut: apply soft correction
%   - Correction strategy: reduce chroma while preserving hue and lightness
%   - Iterative reduction until in-gamut
%   - Diagnostics flag corrected colors

% Algorithm placeholder:
% \begin{algorithm}
%   \caption{Gamut Correction}
%   \label{alg:gamut}
% \end{algorithm}

% ------------------------------------------------------------------------------
% 8.4 Hue Preservation
% ------------------------------------------------------------------------------
\section{Hue Preservation}
\label{sec:hue-preservation}

% [To be written: ~200 words]
% Content: Priority of hue preservation in correction
% Key points:
%   - Hue is most important for color identity
%   - Correction reduces chroma first, then lightness if needed
%   - Never rotates hue to fit gamut
%   - Maintains journey coherence even after correction
%   - Trade-off: may lose saturation but keeps color character
