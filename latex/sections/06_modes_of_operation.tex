% ==============================================================================
% Section 6: Modes of Operation
% ==============================================================================
% Primary Source: 11_modes_of_operation.md
% Target Length: 500-1,000 words
% Dependencies: §3 (journey construction), §4 (distance constraints)
% ==============================================================================

% ------------------------------------------------------------------------------
% 6.1 Journey Mode
% ------------------------------------------------------------------------------
\section{Journey Mode}
\label{sec:journey-mode}

% [To be written: ~250 words]
% Content: Smooth perceptual transitions (default mode)
% Key points:
%   - Default mode of operation
%   - Optimizes for smooth transitions between adjacent colors
%   - Respects Δ_max constraint (no jarring jumps)
%   - Output colors form coherent perceptual sequence
%   - Best for: gradients, timelines, sequential data

% ------------------------------------------------------------------------------
% 6.2 Categorical Mode
% ------------------------------------------------------------------------------
\section{Categorical Mode}
\label{sec:categorical-mode}

% [To be written: ~250 words]
% Content: Maximum distinction between colors
% Key points:
%   - Alternative mode for categorical data
%   - Maximizes perceptual distance between ALL colors (not just adjacent)
%   - Relaxes Δ_max constraint
%   - May sacrifice smoothness for distinction
%   - Best for: legends, categories, labels

% ------------------------------------------------------------------------------
% 6.3 Perceptual Velocity
% ------------------------------------------------------------------------------
\section{Perceptual Velocity}
\label{sec:perceptual-velocity}

% [To be written: ~250 words]
% Content: Rate of perceptual change along the journey
% Key points:
%   - Velocity = Δ_perceptual / Δ_index
%   - Journey Mode: aims for constant velocity (uniform change)
%   - Categorical Mode: velocity is irrelevant (order doesn't matter)
%   - Smoothness parameter affects velocity consistency
%   - Useful diagnostic for evaluating palette quality

% ------------------------------------------------------------------------------
% 6.4 Mode Selection Guidelines
% ------------------------------------------------------------------------------
\section{Mode Selection Guidelines}
\label{sec:mode-selection}

% [To be written: ~200 words]
% Content: Decision tree for choosing mode
% Key points:
%   - Use Journey Mode when order matters (sequential data, animation)
%   - Use Categorical Mode when distinction matters (labels, categories)
%   - Hybrid approaches possible via post-processing
%   - Mode is a top-level API parameter

% Decision tree or table placeholder:
% \begin{table}[h]
%   \centering
%   \caption{Mode Selection Guide}
%   \label{tab:mode-selection}
% \end{table}
