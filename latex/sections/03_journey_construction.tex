% ==============================================================================
% Section 3: Journey Construction
% ==============================================================================
% Primary Source: 02_journey_construction.md
% Target Length: 1,500-2,000 words
% Dependencies: §2 (OKLab coordinates for path construction)
% ==============================================================================

% ------------------------------------------------------------------------------
% 3.1 The Journey Metaphor
% ------------------------------------------------------------------------------
\section{The Journey Metaphor}
\label{sec:journey-metaphor}

% [To be written: ~250 words]
% Content: Explain the "journey" as a constructive device
% Key points:
%   - Colors generated "as if" traveling through perceptual space
%   - Journey provides coherence and natural ordering
%   - NOT prescriptive: users can use colors however they want
%   - Internal construction device, not exposed API concept

% ------------------------------------------------------------------------------
% 3.2 Anchor Colors
% ------------------------------------------------------------------------------
\section{Anchor Colors}
\label{sec:anchors}

% [To be written: ~300 words]
% Content: The anchor system (1-5 colors)
% Key points:
%   - Anchors are user-specified colors that MUST appear in output
%   - Support 1-5 anchors (practical limit for coherent journeys)
%   - Anchors define waypoints the journey must pass through
%   - Position in output guaranteed (not just "close to")

% ------------------------------------------------------------------------------
% 3.3 Single-Anchor Expansion
% ------------------------------------------------------------------------------
\section{Single-Anchor Expansion}
\label{sec:single-anchor}

% [To be written: ~400 words]
% Content: The "mood expansion" algorithm for single anchors
% Key points:
%   - Critical innovation: generate palette from ONE color
%   - Expand in perceptually-coherent directions
%   - Use anchor's character (hue, lightness) to determine expansion
%   - Gray anchors: expand along lightness axis
%   - Chromatic anchors: expand in complementary/analogous directions
%   - Style controls influence expansion direction and extent

% ------------------------------------------------------------------------------
% 3.4 Multi-Anchor Paths
% ------------------------------------------------------------------------------
\section{Multi-Anchor Paths}
\label{sec:multi-anchor}

% [To be written: ~300 words]
% Content: Constructing paths through multiple anchors
% Key points:
%   - 2 anchors: single segment between them
%   - 3+ anchors: piecewise path with smooth connections
%   - Order matters: anchors visited in sequence
%   - Path respects anchor positions in output array

% ------------------------------------------------------------------------------
% 3.5 Bézier Curve Construction
% ------------------------------------------------------------------------------
\section{Bézier Curve Construction}
\label{sec:bezier}

% [To be written: ~400 words]
% Content: Cubic Bézier curves with C¹ continuity
% Key points:
%   - Each segment is cubic Bézier in OKLab space
%   - Control points determined by anchor positions and style params
%   - C¹ continuity at anchor points (smooth tangent)
%   - Intensity parameter scales control point distance

% Bézier formula placeholder:
% \begin{equation}
%   B(t) = (1-t)^3 P_0 + 3(1-t)^2 t P_1 + 3(1-t) t^2 P_2 + t^3 P_3
%   \label{eq:bezier}
% \end{equation}

% ------------------------------------------------------------------------------
% 3.6 Arc-Length Parameterization
% ------------------------------------------------------------------------------
\section{Arc-Length Parameterization}
\label{sec:arc-length}

% [To be written: ~300 words]
% Content: Uniform sampling via arc-length parameterization
% Key points:
%   - Parameter t does not give uniform spacing
%   - Arc-length parameterization: equal t steps = equal distance
%   - Numerical integration to build arc-length table
%   - Enables perceptually-uniform sampling along curve
