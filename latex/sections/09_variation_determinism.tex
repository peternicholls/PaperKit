% ==============================================================================
% Section 9: Variation and Determinism
% ==============================================================================
% Primary Source: 06_variation_determinism.md
% Target Length: 400-600 words
% Dependencies: None (cross-cutting concern)
% ==============================================================================

% ------------------------------------------------------------------------------
% Chapter Introduction
% ------------------------------------------------------------------------------
The preceding chapters define the engine's functional behaviour: how it constructs paths, applies constraints, responds to style controls, handles loops, and manages gamut boundaries. This chapter addresses a cross-cutting concern that underlies all of these capabilities: the requirement for deterministic, reproducible output---and the controlled variation layer that enables exploration without sacrificing predictability.

% ------------------------------------------------------------------------------
% 9.1 Determinism Requirement
% ------------------------------------------------------------------------------
\section{Determinism Requirement}
\label{sec:determinism}

The Color Journey Engine \emph{must} be programmatically deterministic. Given the same inputs, it \emph{must} produce the same outputs, up to small numerical differences ($\sim$0.2\%) arising from floating-point arithmetic and platform differences. This requirement enables the ``config as ID'' pattern central to the API philosophy (\S\ref{sec:api-philosophy}).

This is a hard requirement, not a preference:

\begin{itemize}
    \item \textbf{Debugging} --- Developers must be able to reproduce bugs
    \item \textbf{Testing} --- Automated tests need predictable outputs
    \item \textbf{Sharing} --- Configurations can be shared as compact seeds/parameters
    \item \textbf{Version control} --- Generated assets can be diffed meaningfully
    \item \textbf{Caching} --- Identical inputs can be cached without recomputation
\end{itemize}

For identical output, the following must match: anchor colours (same order and representation), all configuration parameters, variation mode and seed (if used), and implementation version.

No source of non-determinism may affect palette generation: system time, thread scheduling, external state, true random numbers, hash table iteration order, and uninitialised memory are all prohibited.

% ------------------------------------------------------------------------------
% 9.2 Controlled Variation
% ------------------------------------------------------------------------------
\section{Controlled Variation}
\label{sec:variation}

While determinism is required, users may want to generate alternatives, add organic texture, or explore nearby palettes. The \textbf{variation layer} provides this through controlled, seeded perturbations.

\begin{center}
\begin{tabular}{ll}
\toprule
\textbf{Mode} & \textbf{Effect} \\
\midrule
\texttt{off} & No variation; pure interpolation \\
\texttt{subtle} & Small perturbations; $\Delta E < 1$ \\
\texttt{noticeable} & Moderate perturbations; $\Delta E$ 1--2 \\
\texttt{extreme} & Large perturbations; $\Delta E$ 2--4 \\
\bottomrule
\end{tabular}
\end{center}

Variation modes are categorical rather than continuous, preventing the paralysis of infinite choice and providing meaningful distinct options.

Variation adds small offsets to interpolated colours in OKLab space:
\begin{align}
    L' &= L + \delta_L \\
    a' &= a + \delta_a \\
    b' &= b + \delta_b
\end{align}

where perturbations are drawn from a Gaussian distribution with standard deviation scaled by mode.

Anchors are \emph{never} perturbed---variation applies only to interpolated colours. This ensures user-specified key colours remain exact (\S\ref{sec:anchors}). Perturbations taper near anchors (scaling by $\sin(\pi t)$) for smooth transitions.

% ------------------------------------------------------------------------------
% 9.3 Seed Handling
% ------------------------------------------------------------------------------
\section{Seed Handling}
\label{sec:seeds}

The variation layer uses a \textbf{pseudo-random number generator (PRNG)} with these properties~\cite{knuth1997}:

\begin{itemize}
    \item \textbf{Seeded} --- Initialised from user-provided seed
    \item \textbf{Deterministic} --- Same seed produces same sequence
    \item \textbf{Well-distributed} --- Numbers uniformly distributed
    \item \textbf{Specified algorithm} --- Implementation uses documented algorithm (e.g., xoshiro256**~\cite{blackman2018})
\end{itemize}

\begin{center}
\begin{tabular}{ll}
\toprule
\textbf{Seed Value} & \textbf{Behaviour} \\
\midrule
Explicit integer & Use that seed; variation is implicitly enabled \\
\texttt{null} or omitted & Variation disabled; pure interpolation \\
\bottomrule
\end{tabular}
\end{center}

\textit{API clarification:} Providing a seed value implicitly enables variation---there is no separate \texttt{variation: true} toggle. The presence of a seed is sufficient to activate the variation layer. If you want deterministic variation, provide a seed; if you want pure interpolation without variation, omit the seed entirely.

If variation is requested via a seed, the engine uses that seed for the PRNG. The seed value 0 is valid and distinct from omitting the seed.

The order in which random numbers are consumed (per-color L, a, b perturbations) is fixed and documented---changing advancement order would change outputs for existing seeds.

% ------------------------------------------------------------------------------
% 9.4 Reproducibility
% ------------------------------------------------------------------------------
\section{Reproducibility}
\label{sec:reproducibility}

With determinism, a configuration object serves as a unique identifier for a specific palette. This enables the ``config as ID'' pattern:

\begin{itemize}
    \item Store the configuration instead of the full palette
    \item Transmit compactly between systems
    \item Regenerate on any conforming implementation
\end{itemize}

Users can explore alternatives by iterating seeds: \texttt{seed=1} produces variant A, \texttt{seed=2} produces variant B, and so on. Each seed deterministically produces a unique but related palette.

Full reproducibility is guaranteed within the same engine version. Version changes may affect output (documented as breaking changes in semantic versioning). Recommended practice: store full configuration with generated palettes for future reference.
