% ==============================================================================
% Section 2: Perceptual Color Foundations
% ==============================================================================
% Primary Source: 01_perceptual_color_foundations.md
% Target Length: 1,000-1,500 words
% Dependencies: None (foundational section)
% ==============================================================================

% ------------------------------------------------------------------------------
% 2.1 The Perceptual Uniformity Requirement
% ------------------------------------------------------------------------------
\section{The Perceptual Uniformity Requirement}
\label{sec:perceptual-uniformity}

% [To be written: ~300 words]
% Content: Why perceptual uniformity matters for palette generation
% Key points:
%   - Mathematical distance should correspond to perceived difference
%   - RGB/HSL are not perceptually uniform
%   - Equal steps in RGB ≠ equal visual steps
%   - Foundation for meaningful constraints (JND, Δ_min, Δ_max)

% ------------------------------------------------------------------------------
% 2.2 OKLab Color Space
% ------------------------------------------------------------------------------
\section{OKLab Color Space}
\label{sec:oklab}

% [To be written: ~400 words]
% Content: OKLab derivation, properties, coordinate meanings
% Key points:
%   - Designed by Björn Ottosson (2020) for perceptual uniformity
%   - L: lightness (0 = black, 1 = white)
%   - a: green-red axis
%   - b: blue-yellow axis
%   - Euclidean distance approximates perceptual difference
%   - Advantages over CIELAB for practical use

% Mathematical definition placeholder:
% \begin{equation}
%   \Delta E_{OK} = \sqrt{(L_2 - L_1)^2 + (a_2 - a_1)^2 + (b_2 - b_1)^2}
%   \label{eq:oklab-distance}
% \end{equation}

% ------------------------------------------------------------------------------
% 2.3 Cartesian vs. Cylindrical Coordinates
% ------------------------------------------------------------------------------
\section{Cartesian vs. Cylindrical Coordinates}
\label{sec:oklch}

% [To be written: ~300 words]
% Content: OKLCh cylindrical form and when to use each
% Key points:
%   - OKLCh: L (lightness), C (chroma), h (hue angle)
%   - C = √(a² + b²), h = atan2(b, a)
%   - Cartesian (Lab): better for interpolation, distance
%   - Cylindrical (LCh): better for hue-preserving operations
%   - Engine uses both internally for different operations

% ------------------------------------------------------------------------------
% 2.4 Color Space Conversion
% ------------------------------------------------------------------------------
\section{Color Space Conversion}
\label{sec:conversion}

% [To be written: ~300 words]
% Content: Conversion pipeline sRGB → linear → OKLab
% Key points:
%   - sRGB gamma expansion to linear RGB
%   - Linear RGB to OKLab via matrix transformation
%   - Inverse for output (OKLab → linear → sRGB)
%   - Precision considerations

% Conversion matrix placeholder:
% \begin{equation}
%   \begin{bmatrix} L \\ a \\ b \end{bmatrix} = M_{OKLab} \cdot 
%   \begin{bmatrix} R_{linear} \\ G_{linear} \\ B_{linear} \end{bmatrix}
%   \label{eq:oklab-matrix}
% \end{equation}
