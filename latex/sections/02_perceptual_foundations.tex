% ==============================================================================
% Section 2: Perceptual Color Foundations
% ==============================================================================
% Primary Source: 01_perceptual_color_foundations.md
% Target Length: 1,000-1,500 words
% Dependencies: None (foundational section)
% ==============================================================================

% ------------------------------------------------------------------------------
% 2.1 The Perceptual Uniformity Requirement
% ------------------------------------------------------------------------------
\section{The Perceptual Uniformity Requirement}
\label{sec:perceptual-uniformity}

Human vision perceives color differences nonlinearly. A linear blend in RGB space does not produce a linear perceptual transition---interpolating between saturated blue and yellow often produces washed-out grays or unexpected hues at the midpoint, because the intermediate colors pass through an unsaturated region of color space.

Similarly, equal numerical changes in HSV or HSL frequently produce uneven perceptual steps. Some hue regions appear to change faster than others; lightness and saturation interact in ways the coordinate system does not capture.

To address this fundamental mismatch, scientists have developed \emph{perceptually uniform} color spaces where Euclidean distances correspond more closely to perceived color differences \cite{cie1976,fairchild2013}. In such spaces, moving a given distance in any direction produces a change that observers judge to be of similar magnitude.

The engine \emph{must} work in a perceptually uniform space so that equal steps along the journey correspond to equal perceived differences. This is non-negotiable for achieving smooth, natural color transitions.

% ------------------------------------------------------------------------------
% 2.2 OKLab Color Space
% ------------------------------------------------------------------------------
\section{OKLab Color Space}
\label{sec:oklab}

OKLab, introduced by Björn Ottosson in 2020 \cite{ottosson2020}, is a modern perceptual color space designed to predict human color perception accurately while remaining computationally efficient. It has gained rapid adoption, appearing in CSS Color Level 4 \cite{csscolor4} and various creative tools.

Compared to alternatives \cite{mahy1994,safdar2017}, OKLab offers an excellent balance. The following comparison synthesizes Ottosson's original analysis~\cite{ottosson2020} with established color science literature~\cite{fairchild2013}:

\begin{center}
\begin{tabular}{lcccc}
\toprule
\textbf{Color Space} & \textbf{Uniformity} & \textbf{Cost} & \textbf{Stability} & \textbf{Selected} \\
\midrule
sRGB & Poor & Low & Good & No \\
HSV/HSL & Poor & Low & Poor & No \\
CIELAB (1976) & Moderate & Moderate & Poor at extremes & No \\
CAM16-UCS & Excellent & High & Good & No \\
\textbf{OKLab} & \textbf{Excellent} & \textbf{Low} & \textbf{Excellent} & \textbf{Yes} \\
\bottomrule
\end{tabular}
\end{center}

\textit{Uniformity} refers to how closely Euclidean distances match perceived differences; \textit{Cost} measures computational complexity (matrix operations vs.\ iterative solving); \textit{Stability} indicates behavior at gamut boundaries and extreme lightness values. OKLab achieves CAM16-level uniformity~\cite{safdar2017} with CIELAB-level computational cost---a key factor for real-time applications.

A color in OKLab is represented by three coordinates $(L, a, b)$:

\begin{description}
    \item[$L$ --- Perceived lightness] ranges from 0 (perceptual black) to 1 (brightest white). Equal differences in $L$ correspond to equal perceived brightness changes.
    
    \item[$a$ --- Green--red opponent axis] with positive values indicating reddish tints and negative values indicating greenish tints. Zero is achromatic along this axis.
    
    \item[$b$ --- Blue--yellow opponent axis] with positive values indicating yellowish tints and negative values indicating bluish tints. Zero is achromatic along this axis.
\end{description}

The $L$, $a$, $b$ axes are designed to be \emph{approximately orthogonal in perception}. Altering $L$ alone should not significantly change perceived hue or chroma; altering $a$ or $b$ primarily affects hue and saturation without changing perceived lightness.

The perceptual distance between two colors is computed as Euclidean distance:

\begin{equation}
    \Delta E_{OK} = \sqrt{(L_2 - L_1)^2 + (a_2 - a_1)^2 + (b_2 - b_1)^2}
    \label{eq:oklab-distance}
\end{equation}

Ottosson designed OKLab such that a $\Delta E$ of approximately 1.0 corresponds to a just-noticeable difference (JND) for an average observer under standard viewing conditions~\cite{ottosson2020}. This design goal aligns with the broader color science literature on JND thresholds~\cite{luo2001,fairchild2013}, though direct perceptual validation studies specific to OKLab remain limited. The 1.0 threshold is a design target rather than an empirically-validated constant across all viewing conditions.

% ------------------------------------------------------------------------------
% 2.3 Cartesian vs. Cylindrical Coordinates
% ------------------------------------------------------------------------------
\section{Cartesian vs.\ Cylindrical Coordinates}
\label{sec:oklch}

For certain operations---particularly hue manipulation---a cylindrical form is more convenient. OKLCh represents the same colors using:

\begin{description}
    \item[$L$ --- Lightness] identical to OKLab
    \item[$C$ --- Chroma] defined as $C = \sqrt{a^2 + b^2}$, representing distance from the neutral axis (colorfulness/saturation)
    \item[$h$ --- Hue angle] defined as $h = \mathrm{atan2}(b, a)$, the angle around the $a$--$b$ plane
\end{description}

The conversion is straightforward:
\begin{align}
    C &= \sqrt{a^2 + b^2} \\
    h &= \mathrm{atan2}(b, a)
\end{align}

The engine uses both representations internally:
\begin{itemize}
    \item \textbf{Cartesian (OKLab)} for distance calculations and Bézier interpolation, where linear operations are meaningful
    \item \textbf{Cylindrical (OKLCh)} for hue-related manipulations such as warmth bias and complementary color calculations
\end{itemize}

This separation allows clean reasoning about lightness dynamics (along $L$) versus chromatic dynamics (in the $a$--$b$ plane or around $h$).

% ------------------------------------------------------------------------------
% 2.4 Color Space Conversion
% ------------------------------------------------------------------------------
\section{Color Space Conversion}
\label{sec:conversion}

The transformation from sRGB to OKLab follows a well-defined pipeline:

\begin{enumerate}
    \item \textbf{Linearise sRGB} --- Remove gamma encoding to obtain linear RGB values
    \item \textbf{RGB to XYZ} --- Apply the standard sRGB-to-XYZ matrix
    \item \textbf{XYZ to LMS} --- Apply matrix $M_1$ to approximate cone responses
    \item \textbf{Nonlinear compression} --- Apply cube root: $(l', m', s') = (l^{1/3}, m^{1/3}, s^{1/3})$
    \item \textbf{LMS to Lab} --- Apply matrix $M_2$ to obtain final $(L, a, b)$
\end{enumerate}

The transformation matrices are:

\begin{equation}
    M_1 = \begin{pmatrix}
    0.8189330101 & 0.3618667424 & -0.1288597137 \\
    0.0329845436 & 0.9293118715 & 0.0361456387 \\
    0.0482003018 & 0.2643662691 & 0.6338517070
    \end{pmatrix}
    \label{eq:oklab-m1}
\end{equation}

\begin{equation}
    M_2 = \begin{pmatrix}
    0.2104542553 & 0.7936177850 & -0.0040720468 \\
    1.9779984951 & -2.4285922050 & 0.4505937099 \\
    0.0259040371 & 0.7827717662 & -0.8086757660
    \end{pmatrix}
    \label{eq:oklab-m2}
\end{equation}

Each step models aspects of human vision: $M_1$ approximates L, M, S cone responses; the cube root models nonlinear perceptual compression; $M_2$ extracts lightness as a weighted average and opponent-color signals as cone differences.

\subsection*{Computational Efficiency}

The OKLab conversion is computationally trivial:
\begin{itemize}
    \item \textbf{Per conversion:} approximately 50--100 CPU cycles
    \item \textbf{Operations:} 18 multiplications, 12 additions, 3 cube roots
    \item \textbf{Memory:} no allocations; all register operations
    \item \textbf{Parallelisation:} trivially vectorisable with SIMD
\end{itemize}

This efficiency enables real-time palette generation. The entire forward/inverse pipeline uses fixed, known constant matrices---no iterative solving, no conditionals, no special cases.

\subsection*{Implications for the Engine}

The choice of OKLab has cascading implications:

\begin{enumerate}
    \item \textbf{Linear interpolation works} --- A straight line between two colors in OKLab produces a perceptually smooth gradient
    
    \item \textbf{Distance calculations are meaningful} --- Constraints like $\Delta_{\min}$ and $\Delta_{\max}$ correspond to actual perceived differences
    
    \item \textbf{Dynamics are separable} --- The orthogonality of $L$, $a$, $b$ means lightness can be adjusted independently of chromaticity
    
    \item \textbf{Hue interpolation is stable} --- Unlike HSV, where interpolating across the 0°/360° boundary causes problems, OKLab's Cartesian representation avoids discontinuities
\end{enumerate}

Because OKLab is unbounded---it can represent colors outside any physical display gamut---the engine must include explicit gamut mapping (\S\ref{sec:gamut-problem}).
