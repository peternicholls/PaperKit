% ==============================================================================
% Section 12: Conclusion and Future Directions
% ==============================================================================
% Primary Source: Research synthesis
% Target Length: 400-600 words
% Dependencies: All previous sections
% ==============================================================================

% ------------------------------------------------------------------------------
% 12.1 Summary
% ------------------------------------------------------------------------------
\section{Summary}
\label{sec:summary}

% [To be written: ~200 words]
% Content: Recap of the specification
% Key points:
%   - Color Journey Engine: deterministic, perceptually-uniform palette generator
%   - Built on OKLab color space for perceptual foundation
%   - Journey metaphor provides coherence through Bézier curves
%   - Constraints (JND, Δ_min, Δ_max) ensure quality
%   - Flexible API with presets for common use cases

% ------------------------------------------------------------------------------
% 12.2 Key Contributions
% ------------------------------------------------------------------------------
\section{Key Contributions}
\label{sec:contributions}

% [To be written: ~200 words]
% Content: What this specification contributes
% Key points:
%   - Formal specification for reproducible palette generation
%   - Single-anchor "mood expansion" algorithm
%   - Perceptual constraint framework (JND-based)
%   - Two-layer gamut management strategy
%   - Complete API design with clear responsibilities
%   - Preset system encoding expert knowledge

% ------------------------------------------------------------------------------
% 12.3 Future Directions
% ------------------------------------------------------------------------------
\section{Future Directions}
\label{sec:future}

% [To be written: ~200 words]
% Content: Potential enhancements and open questions
% Key points:
%   - Extended color spaces (Display P3, Rec. 2020)
%   - Accessibility-aware palette generation
%   - Multi-palette harmonization
%   - Real-time/animated palette generation
%   - Machine learning for preset discovery
%   - Integration with design systems
%   - Open questions from research phase
