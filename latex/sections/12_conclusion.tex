% ==============================================================================
% Section 12: Conclusion and Future Directions
% ==============================================================================
% Primary Source: Research synthesis
% Target Length: 400-600 words
% Dependencies: All previous sections
% ==============================================================================

% ------------------------------------------------------------------------------
% 12.1 Summary
% ------------------------------------------------------------------------------
\section{Summary}
\label{sec:summary}

This specification presents the Color Journey Engine, a deterministic system for generating perceptually-uniform colour palettes. The engine addresses a persistent challenge in digital colour tools: how to create multi-colour palettes that feel coherent while respecting the non-linear, device-dependent nature of human colour perception.

The solution rests on three pillars. First, the OKLab colour space provides a perceptually uniform foundation where equal numerical distances correspond to equal perceived differences. Second, the journey metaphor---implemented through cubic Bézier curves with arc-length parameterisation---ensures smooth, coherent transitions between colours. Third, the constraint system ($\Delta_{\min}$, $\Delta_{\max}$, JND-based adaptive sampling) guarantees that generated colours are both distinguishable and aesthetically pleasing.

The result is a pure function that transforms configuration into palette: stateless, deterministic, and predictable.

% ------------------------------------------------------------------------------
% 12.2 Key Contributions
% ------------------------------------------------------------------------------
\section{Key Contributions}
\label{sec:contributions}

This specification makes several contributions to the field of algorithmic colour palette generation:

\begin{enumerate}
    \item \textbf{Formal Specification.} A complete, reproducible specification for palette generation that eliminates the ``magic'' often present in colour tools. Same inputs yield same outputs, always.
    
    \item \textbf{Mood Expansion Algorithm.} A novel approach for generating coherent multi-colour journeys from a single anchor, using style parameters to expand in perceptually meaningful directions.
    
    \item \textbf{Perceptual Constraint Framework.} A JND-based system for ensuring minimum distinguishability while allowing stylistic variation, with clear priority hierarchies for conflict resolution.
    
    \item \textbf{Two-Layer Gamut Management.} A prevention-then-correction strategy that maintains journey character while guaranteeing displayable output, with hue preservation as a priority.
    
    \item \textbf{Clear Responsibility Boundaries.} Explicit delineation between engine core and caller responsibilities, enabling a lean, high-performance core (5.6M colours/second) while supporting extension.
\end{enumerate}

% ------------------------------------------------------------------------------
% 12.3 Future Directions
% ------------------------------------------------------------------------------
\section{Future Directions}
\label{sec:future}

Several directions merit future investigation:

\paragraph{Extended Colour Spaces.} While the current specification targets sRGB output, the OKLab foundation supports wider gamuts. Future versions could target Display P3 or Rec.~2020 while maintaining backward compatibility.

\paragraph{Accessibility Integration.} Though accessibility checking remains a caller responsibility, future work could explore optional WCAG-aware generation modes that constrain palettes to meet contrast requirements.

\paragraph{Multi-Palette Harmonisation.} Design systems often require multiple coordinated palettes (primary, secondary, semantic). Algorithms for generating harmonised palette families present an interesting extension.

\paragraph{Preset Discovery.} The current preset system encodes expert knowledge manually. Machine learning approaches could discover effective presets from successful palette examples.

\paragraph{Real-Time Animation.} The engine's performance enables real-time generation. Formalising smooth interpolation between palettes could support animated colour transitions in interactive applications.

The Color Journey Engine provides a foundation for these extensions while delivering immediate value: deterministic, perceptually-sound palette generation with a clean, predictable API.
