% ==============================================================================
% Section 12: Conclusion and Future Directions
% ==============================================================================
% Primary Source: Research synthesis
% Target Length: 400-600 words
% Dependencies: All previous sections
% ==============================================================================

% ------------------------------------------------------------------------------
% 12.1 Summary
% ------------------------------------------------------------------------------
\section{Summary}
\label{sec:summary}

This specification presents the Color Journey Engine, a deterministic system for generating perceptually-uniform colour palettes. The engine addresses a persistent challenge in digital colour tools: how to create multi-colour palettes that feel coherent while respecting the non-linear, device-dependent nature of human colour perception.

The solution rests on three pillars. First, the OKLab colour space \cite{ottosson2020} provides a perceptually uniform foundation where equal numerical distances correspond to equal perceived differences. Second, the journey metaphor---implemented through cubic Bézier curves \cite{farin2002} with arc-length parameterisation---ensures smooth, coherent transitions between colours. Third, the constraint system ($\Delta_{\min}$, $\Delta_{\max}$, JND-based adaptive sampling) guarantees that generated colours are both distinguishable and aesthetically pleasing.

The result is a pure function that transforms configuration into palette: stateless, deterministic, and predictable. For quick navigation during team discussions, see Appendix~D for a concept map and decision rationale summaries.

% ------------------------------------------------------------------------------
% 12.2 Key Contributions
% ------------------------------------------------------------------------------
\section{Key Contributions}
\label{sec:contributions}

This specification makes several contributions to the practice of algorithmic colour palette generation:

\begin{enumerate}
    \item \textbf{Formal Specification.} A complete, reproducible specification for palette generation that eliminates the ``magic'' often present in colour tools. Same inputs yield same outputs, always.
    
    \item \textbf{Mood Expansion Algorithm.} A novel approach for generating coherent multi-colour journeys from a single anchor, using style parameters to expand in perceptually meaningful directions. To our knowledge, this single-anchor expansion approach has not been formally documented elsewhere.
    
    \item \textbf{Perceptual Constraint Framework.} A system for ensuring minimum distinguishability while allowing stylistic variation. The specific threshold values ($\Delta_{\min} \approx 2.0$, $\Delta_{\max} \approx 5.0$) are design heuristics informed by colour science literature \citep{fairchild2013} and practical experimentation, not empirically-validated constants.
    
    \item \textbf{Two-Layer Gamut Management.} A prevention-then-correction strategy following established gamut mapping principles \citep{morovic2008}, maintaining journey character while guaranteeing displayable output.
    
    \item \textbf{Clear Responsibility Boundaries.} Explicit delineation between engine core and caller responsibilities, enabling a lean, high-performance core while supporting extension.
\end{enumerate}

\subsection*{Limitations and Caveats}

Several aspects of this specification warrant further investigation:

\begin{itemize}
    \item The perceptual weights for velocity calculation (\S\ref{sec:perceptual-velocity}) are design heuristics, not empirically-validated values.
    \item OKLab's JND correspondence ($\Delta E \approx 1.0$) is a design target \citep{ottosson2020}; formal perceptual validation specific to OKLab remains limited.
    \item The constraint thresholds ($\Delta_{\min}$, $\Delta_{\max}$) and subdivision caps represent engineering judgment; alternative values may be appropriate for different use cases.
\end{itemize}

% ------------------------------------------------------------------------------
% 12.3 Future Directions
% ------------------------------------------------------------------------------
\section{Future Directions}
\label{sec:future}

Several directions merit future investigation:

\paragraph{Extended Colour Spaces.} While the current specification targets sRGB output, the OKLab foundation supports wider gamuts. Future versions could target Display P3 or Rec.~2020 while maintaining backward compatibility.

\paragraph{Accessibility Integration.} Though accessibility checking remains a caller responsibility, future work could explore optional WCAG-aware generation modes that constrain palettes to meet contrast requirements.

\paragraph{Multi-Palette Harmonisation.} Design systems often require multiple coordinated palettes (primary, secondary, semantic). Algorithms for generating harmonised palette families present an interesting extension.

\paragraph{Symmetry-Constrained Optimisation.} The current engine constructs journeys through direct path specification. Future work could explore optimisation-based approaches that search for ``optimal'' journeys maximising perceptual distance while satisfying constraints. Recent work on symmetry-constrained search in combinatorial domains \citep{moosbauer2025} suggests that constraining search to symmetric or well-behaved curve families could dramatically reduce optimisation complexity---a principle that could apply to journey curve optimisation.

\paragraph{Preset Discovery.} The current preset system encodes expert knowledge manually. Machine learning approaches could discover effective presets from successful palette examples.

\paragraph{Real-Time Animation.} The engine's performance enables real-time generation. Formalising smooth interpolation between palettes could support animated colour transitions in interactive applications.

The Color Journey Engine provides a foundation for these extensions while delivering immediate value: deterministic, perceptually-sound palette generation with a clean, predictable API.
