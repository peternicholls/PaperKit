% ==============================================================================
% Section 4: Perceptual Constraints
% ==============================================================================
% Primary Source: 03_perceptual_constraints.md
% Target Length: 500-1,000 words
% Dependencies: §2 (perceptual distance), §3 (path sampling)
% ==============================================================================

% ------------------------------------------------------------------------------
% 4.1 Just-Noticeable Difference
% ------------------------------------------------------------------------------
\section{Just-Noticeable Difference}
\label{sec:jnd}

The \textbf{Just Noticeable Difference (JND)} is the smallest colour change that an average human observer can detect under standard viewing conditions. This concept originates from Weber's and Fechner's foundational psychophysics work and has been extensively studied in colour science~\cite{fairchild2013}.

In OKLab:
\begin{itemize}
    \item Theoretical JND threshold: $\Delta E \approx 1.0$ unit (OKLab's design target~\cite{ottosson2020})
    \item Practical threshold: $\Delta E \approx 2.0$ units (accounting for real-world variability)
\end{itemize}

The practical threshold of 2.0 units is derived from applying a safety margin to the theoretical JND. Colour science literature consistently shows that laboratory JND values (measured under controlled illumination, neutral adaptation, and foveal viewing) underestimate the threshold needed for reliable discrimination in real applications~\cite{fairchild2013,luo2001}. Factors including ambient lighting variation, observer differences, display calibration, and peripheral viewing all increase the effective threshold. The 2$\times$ multiplier is a conservative engineering choice, not an empirically-validated constant.

\begin{center}
\begin{tabular}{ll}
\toprule
$\Delta E$ \textbf{Range} & \textbf{Perceptual Interpretation} \\
\midrule
$< 1.0$ & Imperceptible to most observers \\
$1.0 - 2.0$ & Barely perceptible, subtle \\
$2.0 - 3.0$ & Noticeable, clear difference \\
$3.0 - 5.0$ & Obvious difference, still smooth \\
$> 5.0$ & Pronounced difference, may feel like a ``step'' \\
\bottomrule
\end{tabular}
\end{center}

\textit{Note: This table represents design guidance derived from Fairchild~\cite{fairchild2013} and practical experience, not a universal standard. Thresholds vary significantly by application context.}

% ------------------------------------------------------------------------------
% 4.2 Δ_min Constraint (Distinguishability)
% ------------------------------------------------------------------------------
\section{$\Delta_{\min}$ Constraint (Distinguishability)}
\label{sec:delta-min}

\begin{equation}
    \Delta_{\min} \approx 2.0
\end{equation}

This is the \textbf{minimum allowed perceptual distance} between any two adjacent swatches.

\textbf{Why 2.0?} This value is set at the practical JND threshold (\S\ref{sec:jnd}) to ensure that every colour step is reliably perceptible across typical viewing conditions. Setting $\Delta_{\min}$ at 1.0 (the theoretical JND) would produce steps that are only \emph{sometimes} distinguishable; the 2.0 threshold provides margin for the variability inherent in real-world colour perception~\cite{fairchild2013}. There is no value in generating colours that viewers cannot reliably distinguish---it wastes palette capacity and creates the impression of redundancy.

\subsection*{Enforcement}

If two anchors are closer than $\Delta_{\min}$:
\begin{itemize}
    \item \textbf{Collapse case:} The engine may treat them as essentially the same anchor, producing no intermediate swatches
    \item \textbf{Skip intermediates:} No interpolation is performed; the journey jumps directly from one to the other
\end{itemize}

When anchors are extremely close ($\Delta < \Delta_{\min}$), the engine does not attempt to create meaningless intermediate colours.

% ------------------------------------------------------------------------------
% 4.3 Δ_max Constraint (Coherence)
% ------------------------------------------------------------------------------
\section{$\Delta_{\max}$ Constraint (Coherence)}
\label{sec:delta-max}

\begin{equation}
    \Delta_{\max} \approx 5.0
\end{equation}

This is the \textbf{maximum allowed perceptual distance} between any two adjacent swatches.

\textbf{Why 5.0?} This value represents the upper bound of ``comfortable'' colour transitions---steps large enough to be clearly distinct but not so large as to feel discontinuous. The 5.0 threshold emerges from the JND interpretation table (\S\ref{sec:jnd}): differences above $\sim$5 units cross from ``obvious but smooth'' into ``pronounced step'' territory. This is a design heuristic informed by practical experimentation with generated palettes, not a value derived from formal perceptual studies. Alternative implementations may choose different thresholds based on their aesthetic goals.

\subsection*{Enforcement}

If a segment between anchors has total distance $D > \Delta_{\max}$:

\begin{enumerate}
    \item \textbf{Subdivide:} The segment is divided into $n$ sub-segments where $n = \lceil D / \Delta_{\max} \rceil$
    \item \textbf{Insert intermediates:} $n-1$ intermediate swatches are inserted
    \item \textbf{Cap subdivisions:} $n \leq 5$ to prevent excessive palette length
\end{enumerate}

\textbf{Why cap at 5?} This limit prevents runaway palette growth when anchors are extremely far apart in colour space. The value 5 is chosen to align with the anchor count limit (\S\ref{sec:anchors})---maintaining cognitive tractability. With at most 5 intermediates between any two anchors, users can reason about the palette structure. If anchors are extremely far apart ($D > 25$ units), each step will exceed $\Delta_{\max}$, which is acceptable: the user has explicitly requested a large colour jump by their anchor selection.

% ------------------------------------------------------------------------------
% 4.4 Adaptive Sampling
% ------------------------------------------------------------------------------
\section{Adaptive Sampling}
\label{sec:adaptive-sampling}

Given anchors $A$ and $B$ with perceptual distance $D$, the adaptive sampling algorithm:

\begin{enumerate}
    \item If $D < \Delta_{\min}$: append only $B$ to output (skip intermediates)
    \item Otherwise: compute $n = \min(5, \lceil D / \Delta_{\max} \rceil)$
    \item For $i = 1$ to $n$: interpolate at $t = i/n$ and append to output
\end{enumerate}

Each segment has length $D/n$, guaranteeing:
\begin{itemize}
    \item Each step $\leq \Delta_{\max}$
    \item Each step $\geq \Delta_{\min}$ (approximately, given the constraints)
\end{itemize}

\begin{center}
\begin{tabular}{ccc}
\toprule
\textbf{Total Distance $D$} & \textbf{Segments $n$} & \textbf{Step Size $D/n$} \\
\midrule
3.0 & 1 & 3.0 \\
7.0 & 2 & 3.5 \\
12.0 & 3 & 4.0 \\
20.0 & 4 & 5.0 \\
30.0 & 5 (capped) & 6.0 \\
\bottomrule
\end{tabular}
\end{center}

When $D$ is very large and capped at $n=5$, step sizes may exceed $\Delta_{\max}$. This is a trade-off for palette manageability.

% ------------------------------------------------------------------------------
% 4.5 Constraint Enforcement
% ------------------------------------------------------------------------------
\section{Constraint Enforcement}
\label{sec:constraint-enforcement}
\label{sec:constraint-conflicts}

Some anchor configurations make constraints impossible to satisfy fully---two anchors very close together cannot yield many distinguishable steps; requesting 100 swatches between nearby colours is unsatisfiable.

The engine prioritises producing output over strict enforcement:

\begin{enumerate}
    \item \textbf{Best effort:} The engine produces the closest valid palette
    \item \textbf{Diagnostics:} Constraint violations are reported (not thrown as errors)
    \item \textbf{$\Delta_{\min}$ priority:} Distinguishability takes precedence---colours below JND are never output
\end{enumerate}

Distance constraints are evaluated on the actual path (after dynamics and gamut mapping are applied), not the straight-line distance between anchors. If dynamics increase path length, more intermediate swatches may be needed; if gamut clipping shortens effective distances, fewer may suffice.

\subsection*{The Perceptual Comfort Zone}

The ideal step size falls in the range $2.0 \leq \Delta E \leq 5.0$---steps are clearly distinguishable but not jarring. This is the ``sweet spot'' where palettes feel both coherent and varied.
