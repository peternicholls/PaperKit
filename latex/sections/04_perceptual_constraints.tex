% ==============================================================================
% Section 4: Perceptual Constraints
% ==============================================================================
% Primary Source: 03_perceptual_constraints.md
% Target Length: 500-1,000 words
% Dependencies: §2 (perceptual distance), §3 (path sampling)
% ==============================================================================

% ------------------------------------------------------------------------------
% 4.1 Just-Noticeable Difference
% ------------------------------------------------------------------------------
\section{Just-Noticeable Difference}
\label{sec:jnd}

% [To be written: ~200 words]
% Content: JND threshold in OKLab
% Key points:
%   - JND ≈ 2 OKLab units (empirical threshold)
%   - Colors closer than JND may appear identical
%   - Foundation for minimum distance constraint
%   - Context-dependent (viewing conditions affect perception)

% ------------------------------------------------------------------------------
% 4.2 Minimum Distance Constraint (Δ_min)
% ------------------------------------------------------------------------------
\section{Minimum Distance Constraint}
\label{sec:delta-min}

% [To be written: ~200 words]
% Content: Δ_min constraint rationale and value
% Key points:
%   - Δ_min ≈ 2 (at or above JND)
%   - Ensures adjacent colors are distinguishable
%   - Prevents "muddy" palettes with imperceptible differences
%   - May limit maximum palette size for given path

% ------------------------------------------------------------------------------
% 4.3 Maximum Distance Constraint (Δ_max)
% ------------------------------------------------------------------------------
\section{Maximum Distance Constraint}
\label{sec:delta-max}

% [To be written: ~200 words]
% Content: Δ_max constraint rationale and value
% Key points:
%   - Δ_max ≈ 5 (soft upper bound)
%   - Prevents jarring jumps in palette
%   - Maintains journey coherence
%   - Can be relaxed in Categorical Mode

% ------------------------------------------------------------------------------
% 4.4 Adaptive Sampling
% ------------------------------------------------------------------------------
\section{Adaptive Sampling}
\label{sec:adaptive-sampling}

% [To be written: ~250 words]
% Content: Algorithm for sampling within constraints
% Key points:
%   - Start with uniform arc-length samples
%   - Check distances between adjacent samples
%   - If distance < Δ_min: remove intermediate samples
%   - If distance > Δ_max: insert additional samples
%   - Iterate until all constraints satisfied or limit reached

% Algorithm pseudocode placeholder:
% \begin{algorithm}
%   \caption{Adaptive Sampling}
%   \label{alg:adaptive}
% \end{algorithm}

% ------------------------------------------------------------------------------
% 4.5 Constraint Enforcement
% ------------------------------------------------------------------------------
\section{Constraint Enforcement}
\label{sec:constraint-enforcement}

% [To be written: ~150 words]
% Content: What happens when constraints conflict
% Key points:
%   - Requested N may be impossible within constraints
%   - Engine returns closest valid palette
%   - Diagnostics indicate constraint status
%   - Δ_min takes priority (distinguishability paramount)
