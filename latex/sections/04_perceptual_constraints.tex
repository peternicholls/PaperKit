% ==============================================================================
% Section 4: Perceptual Constraints
% ==============================================================================
% Primary Source: 03_perceptual_constraints.md
% Target Length: 500-1,000 words
% Dependencies: §2 (perceptual distance), §3 (path sampling)
% ==============================================================================

% ------------------------------------------------------------------------------
% 4.1 Just-Noticeable Difference
% ------------------------------------------------------------------------------
\section{Just-Noticeable Difference}
\label{sec:jnd}

The \textbf{Just Noticeable Difference (JND)} is the smallest colour change that an average human observer can detect under standard viewing conditions.

In OKLab:
\begin{itemize}
    \item Theoretical JND threshold: $\Delta E \approx 1.0$ unit
    \item Practical threshold: $\Delta E \approx 2.0$ units (accounting for viewing conditions, observer variability, and typical display conditions)
\end{itemize}

\begin{center}
\begin{tabular}{ll}
\toprule
$\Delta E$ \textbf{Range} & \textbf{Perceptual Interpretation} \\
\midrule
$< 1.0$ & Imperceptible to most observers \\
$1.0 - 2.0$ & Barely perceptible, subtle \\
$2.0 - 3.0$ & Noticeable, clear difference \\
$3.0 - 5.0$ & Obvious difference, still smooth \\
$> 5.0$ & Pronounced difference, may feel like a ``step'' \\
\bottomrule
\end{tabular}
\end{center}

These thresholds are context-dependent. Dark-adapted viewing, low-contrast backgrounds, and peripheral vision all affect perception. The practical threshold of 2.0 provides margin for real-world variability.

% ------------------------------------------------------------------------------
% 4.2 Δ_min Constraint (Distinguishability)
% ------------------------------------------------------------------------------
\section{$\Delta_{\min}$ Constraint (Distinguishability)}
\label{sec:delta-min}

\begin{equation}
    \Delta_{\min} \approx 2.0
\end{equation}

This is the \textbf{minimum allowed perceptual distance} between any two adjacent swatches.

Setting $\Delta_{\min}$ at the JND threshold ensures that every colour step is perceptible. There is no value in generating colours that viewers cannot distinguish---it wastes palette capacity and creates the illusion of redundancy.

\subsection*{Enforcement}

If two anchors are closer than $\Delta_{\min}$:
\begin{itemize}
    \item \textbf{Collapse case:} The engine may treat them as essentially the same anchor, producing no intermediate swatches
    \item \textbf{Skip intermediates:} No interpolation is performed; the journey jumps directly from one to the other
\end{itemize}

When anchors are extremely close ($\Delta < \Delta_{\min}$), the engine does not attempt to create meaningless intermediate colours.

% ------------------------------------------------------------------------------
% 4.3 Δ_max Constraint (Coherence)
% ------------------------------------------------------------------------------
\section{$\Delta_{\max}$ Constraint (Coherence)}
\label{sec:delta-max}

\begin{equation}
    \Delta_{\max} \approx 5.0
\end{equation}

This is the \textbf{maximum allowed perceptual distance} between any two adjacent swatches.

Setting $\Delta_{\max}$ at approximately 5 units ensures that no single step is so large that it feels like an abrupt jump. Differences above $\sim$5 are readily noticeable and can break the sense of smooth progression.

\subsection*{Enforcement}

If a segment between anchors has total distance $D > \Delta_{\max}$:

\begin{enumerate}
    \item \textbf{Subdivide:} The segment is divided into $n$ sub-segments where $n = \lceil D / \Delta_{\max} \rceil$
    \item \textbf{Insert intermediates:} $n-1$ intermediate swatches are inserted
    \item \textbf{Cap subdivisions:} $n \leq 5$ to prevent excessive palette length
\end{enumerate}

At most 5 segments (4 intermediate swatches) are created between any two anchors, keeping palettes manageable. If anchors are extremely far apart, each step approaches $\Delta_{\max}$.

% ------------------------------------------------------------------------------
% 4.4 Adaptive Sampling
% ------------------------------------------------------------------------------
\section{Adaptive Sampling}
\label{sec:adaptive-sampling}

Given anchors $A$ and $B$ with perceptual distance $D$, the adaptive sampling algorithm:

\begin{enumerate}
    \item If $D < \Delta_{\min}$: append only $B$ to output (skip intermediates)
    \item Otherwise: compute $n = \min(5, \lceil D / \Delta_{\max} \rceil)$
    \item For $i = 1$ to $n$: interpolate at $t = i/n$ and append to output
\end{enumerate}

Each segment has length $D/n$, guaranteeing:
\begin{itemize}
    \item Each step $\leq \Delta_{\max}$
    \item Each step $\geq \Delta_{\min}$ (approximately, given the constraints)
\end{itemize}

\begin{center}
\begin{tabular}{ccc}
\toprule
\textbf{Total Distance $D$} & \textbf{Segments $n$} & \textbf{Step Size $D/n$} \\
\midrule
3.0 & 1 & 3.0 \\
7.0 & 2 & 3.5 \\
12.0 & 3 & 4.0 \\
20.0 & 4 & 5.0 \\
30.0 & 5 (capped) & 6.0 \\
\bottomrule
\end{tabular}
\end{center}

When $D$ is very large and capped at $n=5$, step sizes may exceed $\Delta_{\max}$. This is a trade-off for palette manageability.

% ------------------------------------------------------------------------------
% 4.5 Constraint Enforcement
% ------------------------------------------------------------------------------
\section{Constraint Enforcement}
\label{sec:constraint-enforcement}

Some anchor configurations make constraints impossible to satisfy fully---two anchors very close together cannot yield many distinguishable steps; requesting 100 swatches between nearby colours is unsatisfiable.

The engine prioritises producing output over strict enforcement:

\begin{enumerate}
    \item \textbf{Best effort:} The engine produces the closest valid palette
    \item \textbf{Diagnostics:} Constraint violations are reported (not thrown as errors)
    \item \textbf{$\Delta_{\min}$ priority:} Distinguishability takes precedence---colours below JND are never output
\end{enumerate}

Distance constraints are evaluated on the actual path (after dynamics and gamut mapping are applied), not the straight-line distance between anchors. If dynamics increase path length, more intermediate swatches may be needed; if gamut clipping shortens effective distances, fewer may suffice.

\subsection*{The Perceptual Comfort Zone}

The ideal step size falls in the range $2.0 \leq \Delta E \leq 5.0$---steps are clearly distinguishable but not jarring. This is the ``sweet spot'' where palettes feel both coherent and varied.
