% ==============================================================================
% Section 11: Caller Responsibilities
% ==============================================================================
% Primary Sources: 09_diagnostics_validation.md, 10_scope_use_cases_presets.md
% Target Length: 400-600 words
% Dependencies: §10 (API contract)
% ==============================================================================

% ------------------------------------------------------------------------------
% 11.1 Input Validation
% ------------------------------------------------------------------------------
\section{Input Validation}
\label{sec:input-validation}

The engine performs basic input validation, but callers should validate inputs before submission:

\begin{itemize}
    \item \textbf{Anchor colours} --- Should be valid colour specifications. Out-of-gamut anchors are accepted but may require correction during palette generation.
    \item \textbf{Count} --- Must be a positive integer. Requests for zero or negative counts return an error.
    \item \textbf{Parameters} --- Should respect documented ranges. The engine clamps out-of-range values but callers should validate to avoid surprises.
    \item \textbf{Presets} --- Unknown preset names return an error.
\end{itemize}

% ------------------------------------------------------------------------------
% 11.2 Engine Guarantees
% ------------------------------------------------------------------------------
\section{Engine Guarantees}
\label{sec:guarantees}

The engine provides the following guarantees:

\begin{enumerate}
    \item \textbf{Output count matches request.} The palette contains exactly the requested number of colours.
    \item \textbf{All outputs are valid sRGB.} Every colour in the palette is displayable on standard monitors.
    \item \textbf{Anchors appear in output.} When anchors are specified, they appear at their designated positions in the palette.
    \item \textbf{Perceptual constraints are honoured.} Adjacent colours differ by at least $\Delta_{\min}$ (where achievable).
    \item \textbf{Determinism is maintained.} Identical inputs produce identical outputs across calls and platforms.
\end{enumerate}

% ------------------------------------------------------------------------------
% 11.3 Caller Responsibilities
% ------------------------------------------------------------------------------
\section{Caller Responsibilities}
\label{sec:caller-duties}

The engine generates palettes; callers are responsible for context-specific concerns:

\paragraph{Accessibility Testing.} The engine does not check WCAG contrast ratios. Callers must verify that text-background pairings meet accessibility requirements for their target compliance level (AA, AAA).

\paragraph{Colour Blindness.} The engine does not simulate or optimise for colour vision deficiencies. Callers concerned with colour-blind accessibility should test generated palettes with appropriate simulation tools.

\paragraph{Context Suitability.} Print and screen have different requirements \citep{poynton2012}. Callers must verify that generated palettes are suitable for their target medium, including:
\begin{itemize}
    \item Colour gamut (sRGB output may not match CMYK print)
    \item Viewing conditions (ambient light, screen calibration)
    \item Substrate (paper type, coating)
\end{itemize}

\paragraph{Application Constraints.} Brand colours, corporate guidelines, and domain-specific requirements are caller concerns. The engine provides building blocks; callers assemble them appropriately.

\paragraph{Reproducibility.} Callers should store the config echo returned by the engine if they need to regenerate palettes later. The engine does not maintain history.

% ------------------------------------------------------------------------------
% 11.4 Error Handling
% ------------------------------------------------------------------------------
\section{Error Handling}
\label{sec:error-handling}

The engine handles errors gracefully:

\begin{itemize}
    \item \textbf{Invalid inputs} --- Return an error with diagnostic message. No partial output is produced.
    \item \textbf{Constraint conflicts} --- Resolved using the priority hierarchy (\S\ref{sec:constraint-conflicts}). Adaptations are documented in diagnostics.
    \item \textbf{Impossible requests} --- The closest valid alternative is returned. For example, requesting 100 visually distinct colours from a narrow hue range will produce fewer distinct steps.
    \item \textbf{No silent failures} --- All constraint violations, gamut corrections, and adaptations appear in the diagnostics output.
\end{itemize}
