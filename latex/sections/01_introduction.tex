% ==============================================================================
% Section 1: Introduction and Scope
% ==============================================================================
% Primary Source: 10_scope_use_cases_presets.md
% Target Length: 500-1,000 words
% ==============================================================================

% ------------------------------------------------------------------------------
% 1.1 Motivation
% ------------------------------------------------------------------------------
\section{Motivation}
\label{sec:motivation}

Every domain that works with colour---user interface design, data visualisation, generative art, interactive media---requires coherent colour palettes. Despite its apparent simplicity, producing such palettes presents subtle challenges. Ad-hoc approaches---linear interpolation in RGB space or arbitrary selections from colour wheels---frequently yield palettes with inconsistent perceptual spacing: some adjacent colours appear nearly identical while others jump dramatically.

The fundamental problem is that commonly-used colour spaces (sRGB, HSL, HSV) are not \emph{perceptually uniform}. Equal numerical steps do not correspond to equal perceived differences. A mathematically elegant gradient in RGB may appear uneven to the human eye, with certain hue regions changing faster than others.

This specification addresses the need for a \emph{principled, reproducible} approach to palette generation---one grounded in perceptual colour science, where mathematical operations correspond meaningfully to visual experience.

\subsection*{Audience and Purpose}

This document serves as an internal technical reference for the engineering team developing and maintaining the C-core Color Journey Engine implementation. It documents design rationale, mathematical foundations, and architectural decisions to enable team alignment---ensuring we can say ``see \S\ref{sec:single-anchor}'' and be immediately on the same page.

The paper prioritises the ``why'' over the ``how'': implementation details are covered in separate technical documentation, while this specification captures the reasoning behind design choices, the perceptual science foundations, and the constraints that shape the system.

% ------------------------------------------------------------------------------
% 1.2 Scope Definition
% ------------------------------------------------------------------------------
\section{Scope Definition}
\label{sec:scope}

The Color Journey Engine produces \textbf{ordered sequences of discrete colour swatches} by constructing perceptually-aware paths through colour space. Given one to five anchor colours and a set of style parameters, the engine generates $N$ swatches such that adjacent colours are perceptually distinguishable yet smoothly related.

\subsection*{What the Engine Does}

\begin{itemize}
    \item Generates deterministic colour palettes using perceptual colour science
    \item Outputs discrete swatch arrays (not continuous functions)
    \item Enforces perceptual constraints ensuring distinguishability
    \item Provides diagnostic information about palette quality
\end{itemize}

\subsection*{What the Engine Does Not Do}

The engine deliberately excludes capabilities that belong to callers or separate systems:

\begin{itemize}
    \item \textbf{Accessibility checking}---WCAG contrast verification is caller responsibility
    \item \textbf{Animation control}---temporal playback logic belongs to the application
    \item \textbf{Colour naming}---a separate knowledge domain
    \item \textbf{Colour management}---ICC profiles and device calibration are external concerns
    \item \textbf{Continuous curve evaluation}---the curve is an internal construction device, not an exposed API
\end{itemize}

% ------------------------------------------------------------------------------
% 1.3 Design Principles
% ------------------------------------------------------------------------------
\section{Design Principles}
\label{sec:design-principles}

Five core principles guide the engine's design:

\paragraph{Perceptual Uniformity as Foundation.}
All design decisions rest on the perceptual uniformity of OKLab colour space \cite{ottosson2020}. Mathematical distances correspond to perceived differences, enabling meaningful constraints and smooth transitions.

\paragraph{Discrete Output, Continuous Thinking.}
The engine constructs continuous paths through colour space internally but outputs discrete swatches. The ``journey'' is a construction metaphor---users receive a finite list of colours, not an interpolation function.

\paragraph{Constructive, Not Prescriptive.}
Colours emerge \emph{as if} travelling a perceptual path, providing coherence and natural ordering. Users may apply the resulting swatches in any manner; the ordering is a construction device, not a usage constraint.

\paragraph{Presets Encode Expertise.}
Presets capture expert knowledge about which parameter combinations achieve specific aesthetic goals---they are not simplifications for novices. A ``Cinematic'' preset encodes design wisdom about dramatic colour arcs.

\paragraph{Graceful Degradation.}
Rather than failing on unusual inputs, the engine adapts. Grey anchors expand along the lightness axis; extreme colours adjust their expansion direction; pathological inputs produce documented fallback behaviour.

% ------------------------------------------------------------------------------
% 1.4 Positioning and Novel Contributions
% ------------------------------------------------------------------------------
\section{Positioning and Novel Contributions}
\label{sec:positioning}

\subsection*{Prior Art: Colour Harmony and Palette Generation}

Traditional palette generation approaches fall into two categories. First, \emph{colour harmony rules}---complementary, triadic, analogous, split-complementary---define geometric relationships on the colour wheel~\cite{itten1961}. These rules have aesthetic merit and deep historical roots in art education, but they operate in perceptually non-uniform spaces (typically HSL or HSV) and provide no guarantees about perceived differences between generated colours.

Second, \emph{interpolation-based approaches} generate colours by blending between endpoints. Most creative tools (CSS gradients, design software) use linear interpolation in RGB or HSL space, producing results that can appear perceptually uneven---muddy midpoints, abrupt hue shifts through grey regions, and inconsistent step sizes~\cite{stone2014}.

\subsection*{The Shift to Perceptual Colour Spaces}

Modern tools increasingly adopt perceptually uniform colour spaces to address these limitations. CIELAB~\cite{cie1976} was the first widely-used perceptual space but exhibits known uniformity issues, particularly in blue-violet regions~\cite{mahy1994}. CIEDE2000~\cite{luo2001} improved colour difference calculation but remains computationally complex.

OKLab, introduced by Björn Ottosson in 2020~\cite{ottosson2020}, represents a significant advance: CAM16-level perceptual uniformity~\cite{safdar2017} with computational efficiency suitable for real-time applications. OKLab has achieved rapid industry adoption, appearing in CSS Color Level 4~\cite{csscolor4}, Figma, and various creative tools. The Color Journey Engine builds on this foundation.

\subsection*{Our Contribution}

The Color Journey Engine introduces a \emph{novel combination} of established techniques:

\begin{itemize}
    \item \textbf{Continuous Bézier paths} through OKLab space~\cite{farin2002}, providing flexible curve shapes
    \item \textbf{Arc-length parameterisation}~\cite{piegl1997} for perceptually uniform sampling
    \item \textbf{Perceptual constraints} ($\Delta_{\min}$, $\Delta_{\max}$) ensuring distinguishability without jarring jumps
    \item \textbf{Single-anchor mood expansion}---to our knowledge, a novel approach---using lightness-direction heuristics to generate coherent palettes from one colour
    \item \textbf{Two-layer gamut management}~\cite{morovic2008} preventing and correcting out-of-gamut colours while preserving hue
\end{itemize}

Our contribution lies not in the individual techniques---OKLab, Bézier curves, arc-length sampling, and gamut mapping are established---but in their integration into a coherent system with explicit perceptual constraints and a deterministic, reproducible API. The resulting implementation achieves 5.6 million colours per second, enabling real-time palette generation.

% ------------------------------------------------------------------------------
% 1.5 Paper Organization
% ------------------------------------------------------------------------------
\section{Paper Organization}
\label{sec:organization}

This specification proceeds as follows:

\begin{itemize}
    \item \textbf{\S\ref{sec:perceptual-uniformity}--\ref{sec:conversion}:} Perceptual colour foundations in OKLab
    \item \textbf{\S\ref{sec:journey-metaphor}--\ref{sec:arc-length}:} Journey construction---anchors, Bézier curves, and sampling
    \item \textbf{\S\ref{sec:jnd}--\ref{sec:constraint-enforcement}:} Perceptual constraint framework
    \item \textbf{\S\ref{sec:temperature}--\ref{sec:mode-selection}:} Style controls and modes of operation
    \item \textbf{\S\ref{sec:loop-open}--\ref{sec:hue-preservation}:} Loop strategies and gamut management
    \item \textbf{\S\ref{sec:determinism}--\ref{sec:error-handling}:} Determinism, API design, and caller responsibilities
    \item \textbf{\S\ref{sec:summary}--\ref{sec:future}:} Summary and future directions
\end{itemize}

Appendices provide preset reference tables, mathematical notation, implementation examples, and a quick-reference guide.
