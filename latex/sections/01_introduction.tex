% ==============================================================================
% Section 1: Introduction and Scope
% ==============================================================================
% Primary Source: 10_scope_use_cases_presets.md
% Target Length: 500-1,000 words
% ==============================================================================

% ------------------------------------------------------------------------------
% 1.1 Motivation
% ------------------------------------------------------------------------------
\section{Motivation}
\label{sec:motivation}

% [To be written: ~200 words]
% Content: Problem statement - need for perceptually-coherent, deterministic palettes
% Key points:
%   - Color palette generation is common in UI, visualization, generative art
%   - Ad-hoc approaches produce inconsistent perceptual spacing
%   - Need for principled, reproducible palette generation

% ------------------------------------------------------------------------------
% 1.2 Scope Definition
% ------------------------------------------------------------------------------
\section{Scope Definition}
\label{sec:scope}

% [To be written: ~250 words]
% Content: What the engine IS vs. ISN'T
% Key points:
%   - IS: Deterministic palette generator using perceptual color science
%   - IS: Discrete swatch output (array of colors)
%   - ISN'T: Color picker, real-time renderer, color management system
%   - ISN'T: Accessibility checker (caller responsibility)

% ------------------------------------------------------------------------------
% 1.3 Design Principles
% ------------------------------------------------------------------------------
\section{Design Principles}
\label{sec:design-principles}

% [To be written: ~300 words]
% Content: Core philosophy guiding the specification
% Key themes:
%   - Perceptual uniformity as foundation
%   - Discrete output, continuous thinking
%   - The journey is constructive, not prescriptive
%   - Presets encode expertise
%   - Graceful edge case handling

% ------------------------------------------------------------------------------
% 1.4 Positioning and Novel Contributions
% ------------------------------------------------------------------------------
\section{Positioning and Novel Contributions}
\label{sec:positioning}

% [To be written: ~300 words]
% Content: Brief "related work" and contribution statement
% Key points:
%   RELATED WORK (2-3 paragraphs):
%   - Traditional palette approaches (harmony rules, geometric relationships)
%   - Modern tools and their limitations (HSL interpolation, no perceptual guarantees)
%   - Industry shift toward perceptual color spaces (OKLab adoption in CSS, Adobe, etc.)
%
%   OUR CONTRIBUTION (1-2 paragraphs):
%   - Journey metaphor: continuous Bézier paths through OKLab, discrete sampling
%   - Key innovations:
%     • Single-anchor mood expansion using lightness direction
%     • Guaranteed anchor appearance (not approximate)
%     • Arc-length parameterization for uniform perceptual spacing
%     • Production-grade performance (5.6M colors/second)
%   - Frame as "introduces novel combination" not "first to do"
%
% Citations to include:
%   - Ottosson (2020) - OKLab
%   - W3C CSS Color 4 - industry validation
%   - Kamermans - Bézier mathematics

% ------------------------------------------------------------------------------
% 1.5 Paper Organization
% ------------------------------------------------------------------------------
\section{Paper Organization}
\label{sec:organization}

% [To be written: ~150 words]
% Content: Roadmap of the paper
% Structure:
%   - §2: Perceptual foundations (OKLab)
%   - §3: Journey construction (anchors, Bézier curves)
%   - §4-6: Constraints, controls, modes
%   - §7-9: Loop strategies, gamut, determinism
%   - §10-11: API design, caller responsibilities
%   - §12: Conclusion
%   - Appendices: Presets, notation, examples, quick reference
