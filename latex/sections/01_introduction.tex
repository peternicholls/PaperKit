% ==============================================================================
% Section 1: Introduction and Scope
% ==============================================================================
% Primary Source: 10_scope_use_cases_presets.md
% Target Length: 500-1,000 words
% ==============================================================================

% ------------------------------------------------------------------------------
% 1.1 Motivation
% ------------------------------------------------------------------------------
\section{Motivation}
\label{sec:motivation}

Color palette generation is a ubiquitous requirement across user interface design, data visualisation, generative art, and interactive media. Yet the seemingly simple task of producing a coherent sequence of colours presents subtle challenges. Ad-hoc approaches---such as linear interpolation in RGB space or arbitrary selections from colour wheels---frequently produce palettes with inconsistent perceptual spacing: some adjacent colours appear nearly identical while others jump dramatically.

The fundamental problem is that most commonly-used colour spaces (sRGB, HSL, HSV) are not \emph{perceptually uniform}. Equal numerical steps in these spaces do not correspond to equal perceived differences. A mathematically elegant gradient in RGB may appear uneven to the human eye, with certain hue regions seeming to change faster than others.

This specification addresses the need for a \emph{principled, reproducible} approach to palette generation---one grounded in perceptual colour science, where mathematical operations correspond meaningfully to visual experience.

% ------------------------------------------------------------------------------
% 1.2 Scope Definition
% ------------------------------------------------------------------------------
\section{Scope Definition}
\label{sec:scope}

The Color Journey Engine produces \textbf{ordered sequences of discrete colour swatches} by constructing perceptually-aware paths through colour space. Given 1--5 anchor colours and a set of style parameters, it generates $N$ swatches such that adjacent colours are perceptually distinguishable yet smoothly related.

\subsection*{What the Engine Does}

\begin{itemize}
    \item Generates deterministic colour palettes using perceptual colour science
    \item Outputs discrete swatch arrays (not continuous functions)
    \item Enforces perceptual constraints ensuring distinguishability
    \item Provides diagnostic information about palette quality
\end{itemize}

\subsection*{What the Engine Does Not Do}

The engine deliberately excludes capabilities that belong to callers or separate systems:

\begin{itemize}
    \item \textbf{Accessibility checking}---WCAG contrast verification is caller responsibility
    \item \textbf{Animation control}---temporal playback logic belongs to the application
    \item \textbf{Colour naming}---a separate knowledge domain
    \item \textbf{Colour management}---ICC profiles and device calibration are external concerns
    \item \textbf{Continuous curve evaluation}---the curve is an internal construction device, not an exposed API
\end{itemize}

% ------------------------------------------------------------------------------
% 1.3 Design Principles
% ------------------------------------------------------------------------------
\section{Design Principles}
\label{sec:design-principles}

Five core principles guide the engine's design:

\paragraph{Perceptual Uniformity as Foundation.}
All decisions are grounded in the perceptual uniformity of OKLab colour space \cite{ottosson2020}. Mathematical distances correspond to perceived differences, enabling meaningful constraints and smooth transitions.

\paragraph{Discrete Output, Continuous Thinking.}
The engine internally constructs continuous paths through colour space, but outputs discrete swatches. The ``journey'' is a construction metaphor---users receive a finite list of colours, not an interpolation function.

\paragraph{The Journey Is Constructive, Not Prescriptive.}
Colours are generated \emph{as if} travelling a perceptual path, which provides coherence and natural ordering. However, users may apply the resulting swatches in any manner; the ordering is a construction device, not a usage constraint.

\paragraph{Presets Encode Expertise.}
Presets are not simplifications for novices---they capture expert knowledge about which parameter combinations work well for specific aesthetic goals. A ``Cinematic'' preset encodes design wisdom about dramatic colour arcs.

\paragraph{Graceful Edge Case Handling.}
Rather than failing on unusual inputs, the engine adapts. Grey anchors expand along the lightness axis. Extreme colours adapt their expansion direction. Pathological inputs produce documented fallback behaviour.

% ------------------------------------------------------------------------------
% 1.4 Positioning and Novel Contributions
% ------------------------------------------------------------------------------
\section{Positioning and Novel Contributions}
\label{sec:positioning}

Traditional palette generation approaches rely on colour harmony rules---complementary, triadic, analogous relationships---applied in HSL or HSV spaces. While these geometric relationships have aesthetic merit, they provide no perceptual guarantees. A ``complementary'' pair in HSV may have vastly different perceived brightness.

Modern tools increasingly adopt perceptually uniform colour spaces. OKLab, introduced by Ottosson in 2020 \cite{ottosson2020}, has gained significant traction, appearing in CSS Color Level 4 \cite{csscolor4} and Adobe creative tools. Yet most implementations use these spaces only for interpolation, without exploiting their properties for systematic palette construction.

The Color Journey Engine introduces a \emph{novel combination} of established techniques:

\begin{itemize}
    \item \textbf{Continuous Bézier paths} through OKLab space, providing flexible curve shapes
    \item \textbf{Arc-length parameterisation} for perceptually uniform sampling
    \item \textbf{Perceptual constraints} ($\Delta_{\min}$, $\Delta_{\max}$) ensuring distinguishability without jarring jumps
    \item \textbf{Single-anchor mood expansion} using lightness-direction heuristics
    \item \textbf{Two-layer gamut management} preventing and correcting out-of-gamut colours
\end{itemize}

We do not claim to be the first to use OKLab, Bézier curves, or perceptual constraints individually. Our contribution is their integration into a coherent, production-grade system achieving 5.6 million colours per second.

% ------------------------------------------------------------------------------
% 1.5 Paper Organization
% ------------------------------------------------------------------------------
\section{Paper Organization}
\label{sec:organization}

This specification proceeds as follows:

\begin{itemize}
    \item \textbf{\S\ref{sec:perceptual-uniformity}--\ref{sec:conversion}} establish the perceptual colour foundations in OKLab
    \item \textbf{\S\ref{sec:journey-metaphor}--\ref{sec:arc-length}} describe journey construction: anchors, Bézier curves, and sampling
    \item \textbf{\S\ref{sec:jnd}--\ref{sec:constraint-enforcement}} define the perceptual constraint framework
    \item \textbf{\S\ref{sec:temperature}--\ref{sec:mode-selection}} cover style controls and modes of operation
    \item \textbf{\S\ref{sec:loop-open}--\ref{sec:hue-preservation}} address loop strategies and gamut management
    \item \textbf{\S\ref{sec:determinism}--\ref{sec:error-handling}} specify determinism, API design, and responsibilities
    \item \textbf{\S\ref{sec:summary}--\ref{sec:future}} summarise contributions and future directions
\end{itemize}

Appendices provide preset reference tables, mathematical notation, implementation examples, and a quick-reference guide.
