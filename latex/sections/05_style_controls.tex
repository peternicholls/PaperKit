% ==============================================================================
% Section 5: User-Facing Style Controls
% ==============================================================================
% Primary Source: 04_dynamic_controls.md
% Target Length: 1,000-1,500 words
% Dependencies: §3 (Bézier control point manipulation)
% ==============================================================================

% ------------------------------------------------------------------------------
% 5.1 Temperature Control
% ------------------------------------------------------------------------------
\section{Temperature Control}
\label{sec:temperature}

% [To be written: ~300 words]
% Content: Warm/cool bias via direction blending
% Key points:
%   - Range: -1.0 (cool) to +1.0 (warm), default 0.0
%   - Biases expansion/interpolation toward warm or cool hues
%   - Warm: shifts toward red/orange/yellow region
%   - Cool: shifts toward blue/cyan region
%   - Implemented via weighted direction vectors in OKLab a-b plane

% Formula placeholder:
% \begin{equation}
%   d_{biased} = d_{neutral} + T \cdot d_{temperature}
%   \label{eq:temperature}
% \end{equation}

% ------------------------------------------------------------------------------
% 5.2 Intensity Control
% ------------------------------------------------------------------------------
\section{Intensity Control}
\label{sec:intensity}

% [To be written: ~300 words]
% Content: Control point scaling for curve drama
% Key points:
%   - Range: 0.0 (minimal) to 1.0 (maximum), default 0.5
%   - Scales Bézier control point distance from anchors
%   - Low intensity: gentle, restrained curves
%   - High intensity: dramatic, sweeping curves
%   - Affects chroma and lightness excursions

% ------------------------------------------------------------------------------
% 5.3 Smoothness Control
% ------------------------------------------------------------------------------
\section{Smoothness Control}
\label{sec:smoothness}

% [To be written: ~300 words]
% Content: Transition character and continuity
% Key points:
%   - Range: 0.0 (sharp) to 1.0 (smooth), default 0.7
%   - Controls transition character between segments
%   - High smoothness: C¹ or C² continuity, gradual changes
%   - Low smoothness: more abrupt transitions at anchors
%   - Affects perceptual velocity consistency

% ------------------------------------------------------------------------------
% 5.4 Secondary Parameters
% ------------------------------------------------------------------------------
\section{Secondary Parameters}
\label{sec:secondary-params}

% [To be written: ~300 words]
% Content: Lightness, chroma, contrast, vibrancy, warmth
% Key points:
%   - Lightness: overall lightness bias (-1 to +1)
%   - Chroma: saturation scaling (0 to 1)
%   - Contrast: lightness range expansion (0 to 1)
%   - Vibrancy: chroma variance (0 to 1)
%   - Warmth: synonym/alias for temperature (API convenience)
%   - Each has sensible default (typically 0.5 or neutral)

% ------------------------------------------------------------------------------
% 5.5 Parameter Interactions
% ------------------------------------------------------------------------------
\section{Parameter Interactions}
\label{sec:param-interactions}

% [To be written: ~200 words]
% Content: How parameters combine and potential conflicts
% Key points:
%   - Parameters are largely orthogonal but not independent
%   - High intensity + low smoothness = potentially jarring
%   - Extreme chroma + gamut limits = correction needed
%   - Presets encode tested combinations
%   - Engine validates ranges but allows edge combinations
