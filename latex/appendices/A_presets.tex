% ==============================================================================
% Appendix A: Preset Reference
% ==============================================================================
% Contains recommended presets with parameter values
% ==============================================================================

\section{Overview}
\label{sec:presets-overview}

Presets encode tested parameter combinations that produce aesthetically coherent results for common use cases. Each preset represents a named configuration that has been validated through practical experimentation and user feedback.

Presets serve three purposes:
\begin{enumerate}
    \item \textbf{Quick Start} --- Provide sensible defaults for common scenarios without requiring parameter knowledge
    \item \textbf{Learning Aid} --- Demonstrate effective parameter combinations as examples
    \item \textbf{Starting Points} --- Serve as baselines for customisation; explicit parameters override preset defaults
\end{enumerate}

When a preset is specified in the API call (\S\ref{sec:api-input}), it expands to a parameter set. Any explicit parameters provided alongside the preset will override the preset's defaults, enabling fine-tuning without starting from scratch.

\section{Preset Table}
\label{sec:preset-table}

The following table provides the complete preset reference with all parameter values. Parameters not listed use engine defaults (\S\ref{sec:api-input}).

\begin{longtable}{p{2.2cm}p{3cm}ccccc}
  \caption{Preset Reference} \label{tab:presets} \\
  \toprule
  \textbf{Preset} & \textbf{Use Case} & \textbf{Temp} & \textbf{Int} & \textbf{Smooth} & \textbf{Mode} & \textbf{Loop} \\
  \midrule
  \endfirsthead
  \multicolumn{7}{c}{{\tablename\ \thetable{} -- continued}} \\
  \toprule
  \textbf{Preset} & \textbf{Use Case} & \textbf{Temp} & \textbf{Int} & \textbf{Smooth} & \textbf{Mode} & \textbf{Loop} \\
  \midrule
  \endhead
  \bottomrule
  \endfoot
  \bottomrule
  \endlastfoot
  
  \texttt{linear} & Simple interpolation & 0.0 & 0.0 & 0.7 & journey & open \\
  \texttt{smooth} & Gentle gradients & 0.0 & 0.3 & 0.9 & journey & open \\
  \texttt{vivid} & High-impact palettes & 0.0 & 0.8 & 0.6 & journey & open \\
  \texttt{muted} & Subdued, subtle & 0.0 & 0.2 & 0.8 & journey & open \\
  \midrule
  \texttt{sunset} & Warm gradients & 0.7 & 0.6 & 0.8 & journey & open \\
  \texttt{sunrise} & Dawn warmth & 0.5 & 0.5 & 0.9 & journey & open \\
  \texttt{ocean} & Cool aquatic tones & $-0.6$ & 0.5 & 0.7 & journey & open \\
  \texttt{forest} & Natural greens & $-0.3$ & 0.4 & 0.8 & journey & open \\
  \texttt{autumn} & Fall foliage & 0.6 & 0.7 & 0.6 & journey & open \\
  \texttt{arctic} & Ice and snow & $-0.8$ & 0.3 & 0.9 & journey & open \\
  \midrule
  \texttt{neon} & Electric, vibrant & 0.0 & 1.0 & 0.5 & journey & closed \\
  \texttt{pastel} & Soft, light & 0.0 & 0.2 & 0.9 & journey & open \\
  \texttt{earth} & Warm naturals & 0.4 & 0.3 & 0.8 & journey & open \\
  \texttt{monochrome} & Single-hue depth & 0.0 & 0.1 & 0.95 & journey & open \\
  \midrule
  \texttt{categorical} & Distinct categories & 0.0 & 0.3 & 0.5 & categorical & open \\
  \texttt{diverging} & Two-pole data & 0.0 & 0.5 & 0.7 & journey & open \\
  \texttt{sequential} & Ordered data & 0.0 & 0.4 & 0.8 & journey & open \\
  \midrule
  \texttt{breathing} & Pulsing animation & 0.0 & 0.4 & 0.95 & journey & pingpong \\
  \texttt{cycling} & Continuous loop & 0.0 & 0.5 & 0.9 & journey & closed \\
  \texttt{evolving} & Progressive shift & 0.0 & 0.5 & 0.8 & journey & phased \\
  
\end{longtable}

\paragraph{Key.} \textbf{Temp} = Temperature ($-1$ to $+1$); \textbf{Int} = Intensity ($0$ to $1$); \textbf{Smooth} = Smoothness ($0$ to $1$); \textbf{Mode} = journey or categorical (\S\ref{sec:journey-mode}, \S\ref{sec:categorical-mode}); \textbf{Loop} = open, closed, pingpong, or phased (\S\ref{sec:loop-open}--\S\ref{sec:loop-phased}).

\section{Preset Descriptions}
\label{sec:preset-descriptions}

\subsection{Basic Presets}

\paragraph{\texttt{linear}} The most basic preset---pure linear interpolation between anchors with no curve enhancement. Useful as a baseline for comparison or when mathematical predictability is paramount. Zero intensity means control points lie exactly on the anchor line (\S\ref{sec:bezier}).

\paragraph{\texttt{smooth}} Gentle, flowing gradients with high smoothness for seamless $C^1$ continuity at anchor junctions. Appropriate for backgrounds, ambient effects, and contexts where the color transition should feel natural and unobtrusive.

\paragraph{\texttt{vivid}} High-intensity curves that create dramatic chromatic excursions between anchors. The path swings through higher-chroma regions, producing bold, attention-grabbing palettes. Best for creative applications where visual impact is desired.

\paragraph{\texttt{muted}} Subdued palettes with low intensity and high smoothness. Produces sophisticated, understated color combinations suitable for professional interfaces, document themes, and contexts requiring visual restraint.

\subsection{Mood-Based Presets}

\paragraph{\texttt{sunset}} Warm temperature bias (+0.7) routes the hue path through reds, oranges, and yellows. High smoothness ensures the gradient feels like a natural sky progression. Ideal for conveying warmth, comfort, or evening atmosphere.

\paragraph{\texttt{sunrise}} Similar to sunset but slightly cooler and softer, capturing the gentler warmth of dawn. Higher smoothness (0.9) creates an ethereal quality.

\paragraph{\texttt{ocean}} Cool temperature bias ($-0.6$) emphasises blues, cyans, and aquatic greens. The path avoids warm hues entirely, creating cohesive underwater or maritime palettes.

\paragraph{\texttt{forest}} Moderate cool bias with natural green emphasis. Lower intensity preserves the organic, muted quality of woodland colors without artificial saturation.

\paragraph{\texttt{autumn}} Warm bias with higher intensity to capture the vibrant oranges, reds, and golds of fall foliage. Slightly lower smoothness allows for more distinct color ``stops'' resembling individual leaves.

\paragraph{\texttt{arctic}} Strong cool bias ($-0.8$) with low intensity and high smoothness for ice, snow, and cold atmospheres. The resulting palettes feel crisp and minimal.

\subsection{Stylistic Presets}

\paragraph{\texttt{neon}} Maximum intensity with closed loop for continuous cycling. Creates electric, high-energy palettes suitable for nightclub aesthetics, gaming interfaces, or attention-grabbing animations.

\paragraph{\texttt{pastel}} Low intensity with very high smoothness produces soft, desaturated palettes. The colors feel gentle and approachable---suitable for children's applications, wellness themes, or light UI backgrounds.

\paragraph{\texttt{earth}} Warm bias with restrained intensity for natural, organic color combinations. Browns, tans, and warm neutrals dominate, suitable for outdoor, craft, or sustainability themes.

\paragraph{\texttt{monochrome}} Minimal intensity (0.1) with near-maximum smoothness for single-hue depth exploration. When combined with a single anchor (\S\ref{sec:single-anchor}), produces sophisticated tonal variations of one color.

\subsection{Data Visualisation Presets}

\paragraph{\texttt{categorical}} Uses Categorical Mode (\S\ref{sec:categorical-mode}) to maximise perceptual distance between all color pairs. Essential for charts, legends, and any context where colors represent distinct categories that must be easily distinguished.

\paragraph{\texttt{diverging}} Journey Mode with balanced parameters, intended for two-anchor use where one anchor represents a negative extreme and the other a positive extreme. The neutral midpoint is automatically generated through interpolation.

\paragraph{\texttt{sequential}} Optimized for ordered data (low-to-high, early-to-late). Smooth progression with moderate intensity creates clear directionality in the palette.

\subsection{Animation Presets}

\paragraph{\texttt{breathing}} Ping-pong loop (\S\ref{sec:loop-pingpong}) with very high smoothness for pulse effects. The color smoothly advances then retreats, creating a ``breathing'' rhythm without any discontinuity at reversal points.

\paragraph{\texttt{cycling}} Closed loop (\S\ref{sec:loop-closed}) for continuous rotation. High smoothness ensures no visible ``seam'' where the loop wraps from end to beginning.

\paragraph{\texttt{evolving}} Phased loop (\S\ref{sec:loop-phased}) for long-running animations that should feel alive. Each cycle shifts slightly, preventing the static feeling of exact repetition.

\section{Preset Selection Guide}
\label{sec:preset-selection}

\begin{center}
\begin{tabular}{p{5.5cm}l}
\toprule
\textbf{If you need...} & \textbf{Try preset} \\
\midrule
Simple, predictable gradient & \texttt{linear} \\
Warm, inviting atmosphere & \texttt{sunset} or \texttt{earth} \\
Cool, calm, professional & \texttt{ocean} or \texttt{arctic} \\
High-energy, attention-grabbing & \texttt{vivid} or \texttt{neon} \\
Soft, gentle, approachable & \texttt{pastel} or \texttt{muted} \\
Chart/graph categories & \texttt{categorical} \\
Heatmap or sequential data & \texttt{sequential} \\
Looping animation & \texttt{cycling} or \texttt{breathing} \\
Tonal variations of one color & \texttt{monochrome} \\
\bottomrule
\end{tabular}
\end{center}
