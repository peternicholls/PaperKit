% Appendix D: Quick Reference for Team Discussions
\section{Quick Reference for Team Discussions}

This appendix provides rapid lookup of key concepts, design decisions, and section references for use during code reviews and design discussions.

\subsection{Concept Map}
\begin{table}[h]
\centering
\begin{tabular}{lll}
\toprule
\textbf{Topic} & \textbf{Section} & \textbf{Key Insight} \\
\midrule
Why OKLab? & \S2.1 & Perceptual uniformity; industry adoption \\
Anchor guarantee & \S3.2 & Anchors appear exactly, not "close to" \\
Mood expansion & \S3.3 & Single-anchor uses lightness direction \\
$\Delta_{min}$ constraint & \S4.2 & JND-based (\~2 $\Delta E$), distinguishability \\
$\Delta_{max}$ constraint & \S4.3 & Ensures coherence, no jarring jumps \\
Loop output & \S7.6 & Flat array (unrolled), no nested structure \\
Gamut correction & \S8.3 & Reduce chroma, preserve hue \\
Determinism & \S9 & Same inputs = same outputs (pure function) \\
Performance & \S10.5 & Microsecond-range, 5.6M colors/second \\
\bottomrule
\end{tabular}
\caption{Quick reference mapping topics to sections}
\end{table}

\subsection{Decision Rationale Summary}
% Summaries of tcolorboxes in main text. Keep concise.
\begin{itemize}
    \item Mood expansion (\S3.3): Single-anchor path uses lightness direction; alternative (naive hue spin) rejected to avoid non-perceptual artifacts.
    \item Loop output (\S7.6): Unrolled arrays prevent nested structures; alternatives (segmented outputs) rejected for caller simplicity.
    \item Gamut handling (\S8.2--\S8.3): Two-layer approach (prevention + correction); hard clipping rejected to preserve hue; HSV fallback rejected to maintain perceptual uniformity.
    \item API scope (\S10): Engine owns core generation; bindings/apps own naming, accessibility, and usage-specific logic.
\end{itemize}

% Note: Ensure \usepackage{booktabs} in preamble for \toprule/\midrule/\bottomrule.
